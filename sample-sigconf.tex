\documentclass[sigconf]{acmart}

\usepackage{booktabs} % For formal tables


% Copyright
%\setcopyright{none}
%\setcopyright{acmcopyright}
%\setcopyright{acmlicensed}
\setcopyright{rightsretained}
%\setcopyright{usgov}
%\setcopyright{usgovmixed}
%\setcopyright{cagov}
%\setcopyright{cagovmixed}


% DOI
\acmDOI{10.475/123_4}

% ISBN
\acmISBN{123-4567-24-567/08/06}

%Conference
\acmConference[SIGIR'17]{ACM SIGIR Conferenceon Research and Development in Information Retrieval}{August 2017}{Tokyo, Japan}
\acmYear{2017}
\copyrightyear{2017}

\acmPrice{15.00}


\begin{document}
\title{SIGIR 2017 Workshop on eCommerce}

\author{Jon Degenhardt}
\affiliation{
	\institution{eBay Inc.}
	\country{USA}
	}


\author{Surya Kallumadi}
\affiliation {
	\institution{Kansas State University}
	\country{USA}
	}

\author{Maarten de Rijke}
\affiliation{
	\institution{University of Amsterdam}
	\country{The Netherlands}
	}


\author{Luo Si}
\affiliation {
	\institution{Alibaba Inc.}
	\country{China}
	}

\author{Andrew Trotman}
\affiliation{
	\institution{University of Otago}
	\country{New Zealand}
	}

\author{Xu Yinghui}
\affiliation{
	\institution{Taobao}
	\country{Japan}
	}

% The default list of authors is too long for headers}
\renewcommand{\shortauthors}{J. Degenhardt et al.}


\begin{abstract}
eCommerce Information Retrieval has received little attention in the
academic literature, yet it is an essential component of some of the
largest web sites (such as eBay, Amazon, Airbnb,  Alibaba, Taobao, Target, Facebook, and others).  SIGIR has for
several years seen sponsorship from these kinds of sites, who clearly value the
importance of research into Information Retrieval.  This
workshop will bring together researchers and practitioners of
eCommerce IR to discuss topics unique to it, to set a research agenda,
and to examine how to build a dataset for research into this fascinating topic.

eCommerce IR is ripe for research and has a unique
set of problems.  For example in eCommerce search there may be no hypertext links
between documents (products); there is a click stream, but more importantly, there
is often a buy stream.  eCommerce problems are wide in scope and range from user interaction
modalities (the kinds of search seen in when buying are different from those
of web-page search (i.e. it is not clear how shopping and buying relate to the standard
web-search interaction models)) through to dynamic updates of a rapidly
changing collection on auction sites, and the experienceness of some products (such as Airbnb bookings).
\end{abstract}

%
% The code below should be generated by the tool at
% http://dl.acm.org/ccs.cfm
% Please copy and paste the code instead of the example below. 
%

\begin{CCSXML}
<ccs2012>
<concept>
<concept_id>10002951.10003317.10003371.10010852</concept_id>
<concept_desc>Information systems~Environment-specific retrieval</concept_desc>
<concept_significance>500</concept_significance>
</concept>
</ccs2012>
\end{CCSXML}

\ccsdesc[500]{Information systems~Environment-specific retrieval}

\keywords{eCommerce, Product Search, Recommendation}


\maketitle

\section{Introduction}

Search, ranking and recommendation have applications ranging from
traditional web search to document databases to vertical search systems.
This workshop explores approaches for search and
recommendations of products.  Although the task is the same as web-page
search (fulfill a user's information need), the way in which this is
achieved is very much different.  On product sites (such as eBay, Flipkart,
Amazon, and Alibaba), the traditional web-page ranking features are either
not present or are present in a different form.

The entities that need to be discovered (the information that fulfills
the need) might be unstructured, associated
with structure, semi-structured, or have facets such as: price,
ratings, title, description, seller location, and so on. 

Domains with such facets raise interesting research challenges such as
a) relevance and ranking functions that take into account the tradeoffs
across various facets with respect to the input query b) recommendations
based on entity similarity c) recommendations based on user location
(e.g. shipping cost), and so on. In the case of eCommerce IR
these challenges require inherent
understanding of product attributes, user behavior, and the query
context. Product sites are characterized by the presence of a dynamic
inventory with a high rate of change and turnover, and a long tail of
query distribution.

Outside of search but still within Information Retrieval, the same
feature in different domains can have radically different meaning.  For
example, in email filtering the presence of ``Ray-Ban''  along with a
price is a strong indication of spam, but within an auction setting
this likely indicates a valid product for sale.  Another example is
natural language translation; company names, product names, and even
product descriptions do not translate well with existing tools.  Similar
problems exist with knowledge graphs that are not customised to match
the product domain.

This workshop brings together researchers and practitioners to
identify a set of core research questions in eCommerce Information Retrieval.
This will include discussion of a research agenda which will serve many purposes. First, collaboration: it will bring the
community together in a way that has never happened before.  Second,
funds: it will help attract research funding to search in this domain.
Third, research: it will help attract researchers and postgraduate
students to eCommerce IR.  Finally, it will help broaden the
definition of information retrieval at conferences such as SIGIR. 

This workshop will also examine the problem of data availability.  As the
purpose of a product site is to make data on entities available, the
same security concerns that plague other search domains may not exist.
However sales and seller information is private and proprietary and
likely to be unavailable.  We expect that the discussion on data will result
in both a proposal to release data that can be put to an eCommerce site, as well
as some tasks that can be examined on that data set.



\section{Themes and Purpose}
This workshop provides a venue for publication and discussion of
Information Retrieval research and ideas as they pertain to eCommerce.
It brings together practitioners and researchers from academia and
industry to discuss the challenges and approaches to search and
recommendation.  A goal is to foster collaboration and discussion in the
broader IR community.  We have an agenda to raise awareness within the
academic community of the problems faced by this domain.

\subsection{Scope}
The workshop relates to all aspects of eCommerce Information Retrieval.
Research topics and challenges that are usually encountered in this domain include:

\begin{itemize}
\item Machine learning techniques such as online learning and deep learning for eCommerce applications
\item Semantic representation for users, products and services \& Semantic understanding of queries
\item Structured data and faceted search, converting unstructured data to its structured form
\item The use of domain specific facets in search and other IR tasks, and how those facets are chosen
\item Temporal dynamics for Search and Recommendation
\item Models for relevance and ranking for multi-faceted entities
\item Deterministic (and other) sorting of results lists (e.g. price low to high including postage)
\item Personalized search and recommendations
\item Inventory display issues (for example: legal, ethical, and spam)
\item Cold start issues
\item Personalization and the use of personal facets such as age, gender, location etc.
\item Indexing and search in a rapidly changing environment (for example, an auction site)
\item Scalability 
\item Diversity in product search and recommendations
\item Strategies for resolving extremely low (or no) recall queries
\item Query intent
\item The use of external features such as reviews and ratings in ranking
\item User interfaces and personalization
\item Reviews and sentiment analysis
\item The use of social signals in ranking and beyond
\item The balance between business requirements and user requirements (revenue vs relevance)
\item Trust
\item Live experimentation Desktop and mobile issues
\item Questions and answering, chatbots for eCommerce
\end{itemize}


\section{Workshop Format}

The workshop will start with an invited talks from  well respected
practitioners (or academics) who are tackling eCommerce Information Retrieval problems.

A call asking not only for research papers, but also for position and opinion papers and posters
has been circulated.  All submitted papers and posters will be single-blind, peer reviewed by an
international program committee of researchers of high repute.  Accepted
papers will be presented with ample time for discussion.  There will be a poster sessio over lunch.

After the thought provoking invited talk and presentations the participants
will break-out into small groups to identify key areas for future
research in product search. Each group identiftying one key research
problem, the challenges and opportunities, and report back to the
workshop on their findings.

The final session of the day will be a panel discussion.  The topic has
not yet been set, but an obvious topic is data availability.
Specifically, what guarantees need to be in place for an organisation
like to makea dump available, and can we meet those guarantees?

The workshop schedule and activities are structured to contain
substantial time for discussion and engagement by all participants.

\section{Participation}
This workshop is open to all interested parties.


\end{document}
