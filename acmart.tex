% This is "sig-alternate.tex" V2.0 May 2012
% This file should be compiled with V2.5 of "sig-alternate.cls" May 2012
%
% This example file demonstrates the use of the 'sig-alternate.cls'
% V2.5 LaTeX2e document class file. It is for those submitting
% articles to ACM Conference Proceedings WHO DO NOT WISH TO
% STRICTLY ADHERE TO THE SIGS (PUBS-BOARD-ENDORSED) STYLE.
% The 'sig-alternate.cls' file will produce a similar-looking,
% albeit, 'tighter' paper resulting in, invariably, fewer pages.
%
% ----------------------------------------------------------------------------------------------------------------
% This .tex file (and associated .cls V2.5) produces:
%       1) The Permission Statement
%       2) The Conference (location) Info information
%       3) The Copyright Line with ACM data
%       4) NO page numbers
%
% as against the acm_proc_article-sp.cls file which
% DOES NOT produce 1) thru' 3) above.
%
% Using 'sig-alternate.cls' you have control, however, from within
% the source .tex file, over both the CopyrightYear
% (defaulted to 200X) and the ACM Copyright Data
% (defaulted to X-XXXXX-XX-X/XX/XX).
% e.g.
% \CopyrightYear{2007} will cause 2007 to appear in the copyright line.
% \crdata{0-12345-67-8/90/12} will cause 0-12345-67-8/90/12 to appear in the copyright line.
%
% ---------------------------------------------------------------------------------------------------------------
% This .tex source is an example which *does* use
% the .bib file (from which the .bbl file % is produced).
% REMEMBER HOWEVER: After having produced the .bbl file,
% and prior to final submission, you *NEED* to 'insert'
% your .bbl file into your source .tex file so as to provide
% ONE 'self-contained' source file.
%
% ================= IF YOU HAVE QUESTIONS =======================
% Questions regarding the SIGS styles, SIGS policies and
% procedures, Conferences etc. should be sent to
% Adrienne Griscti (griscti@acm.org)
%
% Technical questions _only_ to
% Gerald Murray (murray@hq.acm.org)
% ===============================================================
%
% For tracking purposes - this is V2.0 - May 2012

\documentclass{acmart}

\usepackage{color}

\newcommand{\red}[1]{\textcolor{red}{#1}}

\begin{document}
%
% --- Author Metadata here ---
%\conferenceinfo{SIGIR}{2017 Tokyo, Japam}
%\CopyrightYear{2007} % Allows default copyright year (20XX) to be over-ridden - IF NEED BE.
%\crdata{0-12345-67-8/90/01}  % Allows default copyright data (0-89791-88-6/97/05) to be over-ridden - IF NEED BE.
% --- End of Author Metadata ---

\title{SIGIR 2017 Workshop on Product Search and Recommendation}

%
% You need the command \numberofauthors to handle the 'placement
% and alignment' of the authors beneath the title.
%
% For aesthetic reasons, we recommend 'three authors at a time'
% i.e. three 'name/affiliation blocks' be placed beneath the title.
%
% NOTE: You are NOT restricted in how many 'rows' of
% "name/affiliations" may appear. We just ask that you restrict
% the number of 'columns' to three.
%
% Because of the available 'opening page real-estate'
% we ask you to refrain from putting more than six authors
% (two rows with three columns) beneath the article title.
% More than six makes the first-page appear very cluttered indeed.
%
% Use the \alignauthor commands to handle the names
% and affiliations for an 'aesthetic maximum' of six authors.
% Add names, affiliations, addresses for
% the seventh etc. author(s) as the argument for the
% \additionalauthors command.
% These 'additional authors' will be output/set for you
% without further effort on your part as the last section in
% the body of your article BEFORE References or any Appendices.

%\numberofauthors{3} %  in this sample file, there are a *total*
% of EIGHT authors. SIX appear on the 'first-page' (for formatting
% reasons) and the remaining two appear in the \additionalauthors section.
%
%\numberofauthors{5}

\author{
\alignauthor
David Goldberg \\
  \affaddr{eBay}\\
  \affaddr{San Jose}\\
  \affaddr{California}\\
	\email{dgoldberg@ebay.com}
\alignauthor
Andrew Trotman \\
  \affaddr{University of Otago} \\
  \affaddr{Dunedin} \\
  \affaddr{New Zealand} \\
	\email{andrew@cs.otago.ac.nz}
\and
\alignauthor
Xiao Wang \\
  \affaddr{eBay} \\
  \affaddr{San Jose} \\
  \affaddr{California} \\
	\email{xwang2@ebay.com}
\alignauthor
Wei Min \\
  \affaddr{CreditX} \\
  \affaddr{Shanghai} \\
  \affaddr{PRC} \\
\email{vera\_jack312@126.com}
\alignauthor
Zongru Wan \\
  \affaddr{Evolution Labs} \\
  \affaddr{Shanghia} \\
  \affaddr{PRC} \\
	\email{zwan@evolutionlabs.com.cn}
}

\maketitle \begin{abstract}

Product search is a branch of Information Retrieval (IR) that has received little attention in the
academic literature, yet it is an essential component of some of the
largest web sites (such as eBay, Amazon, Airbnb,  Alibaba, Target, Facebook, and others).  SIGIR has for
several years seen sponsorship from these sites, who clearly value the
importance of research into Information Retrieval.  The purpose of this
workshop is to bring together researchers and practitioners of
product search to discuss topics unique to it, to set a research agenda,
and to examine how to build a dataset for research into this fascinating topic.

As a branch of IR, product search is ripe for research and has a unique
set of problems.  For example, in product search there may be no hypertext links
between documents (products); there is a click stream on an eCommerce site, but more importantly, there
is a buy stream.  The problems are wide in scope and range from user interaction
modalities (the kinds of search seen in when buying are different from those
of web-page search (i.e. it is not clear how shopping and buying relate to the standard
web-search interaction models)) through to dynamic updates of a rapidly
changing collection on auction sites, and the experienceness of some products (such as Airbnb bookings).

\end{abstract}
% A category with the (minimum) three required fields
\category{H.3.3}{Information Systems}{Information Storage and
Retrieval}[Information Search and Retrieval]
%A category including the fourth, optional field follows...

\keywords{Product Search, Recommendation, eCommerce}

\section{Introduction}

Search, ranking and recommendation have applications ranging from
traditional web search to document databases to vertical search systems.
In this workshop we will explore approaches for search and
recommendations of products.  Although the task is the same as web-page
search (fulfill a user's information need), the way in which this is
achieved is very much different.  On product sites (such as eBay, Flipkart,
Amazon, and Alibaba), the traditional web-page ranking features are either
not present or are present in a different form.

The entities that need to be discovered (the information that fulfills
the need) might be unstructured, associated
with structure, semi-structured, or have facets such as: price,
ratings, title, description, seller location, and so on. 

Domains with such facets raise interesting research challenges such as
a) relevance and ranking functions that take into account the tradeoffs
across various facets with respect to the input query b) recommendations
based on entity similarity c) recommendations based on user location
(e.g. shipping cost), and so on. In the case of product search and
recommendation these challenges require inherent
understanding of product attributes, user behavior, and the query
context. Product sites are also characterized by the presence of a dynamic
inventory with a high rate of change and turnover, and a long tail of
query distribution.

Outside of search but still within Information Retrieval, the same
feature in different domains can have radically different meaning.  For
example, in email filtering the presence of ``Ray-Ban''  along with a
price is a strong indication of spam, but within an auction setting
this likely indicates a valid product for sale.  Another example is
natural language translation; company names, product names, and even
product descriptions do not translate well with existing tools.  Similar
problems exist with knowledge graphs that are not customised to match
the product domain.

This workshop will bring together researchers and practitioners to
identify a set of core research questions in product search and
recommendation.  This will result in a written research agenda which
will serve many purposes. First, collaboration: it will bring the
community together in a way that has never happened before.  Second,
funds: it will help attract research funding to search in this domain.
Third, research: it will help attract researchers and postgraduate
students to product (and eCommerce) search.  Finally, it will help broaden the
definition of information retrieval at conferences such as SIGIR. 

The workshop will also examine the problem of data availability.  As the
purpose of a product site is to make data on entities available, the
same security concerns that plague other search domains do not exist.
However sales and seller information is private and proprietary and
likely to be unavailable.  We hope that the discussion on data will result
in both a proposal to release data that can be put to an eCommerce site, as well
as some tasks that can be examined on that data set.  Such might include
the reliable clustering of auction items into products for a site such
as eBay.

\subsection{Appropriateness to SIGIR}

The primary theme of the workshop is product search and recommendation
(i.e. Information Retrieval).  We believe that as such it is appropriate
for SIGIR. 

\section{Related Workshops}

We are not aware of any product workshops at SIGIR.  We are, however,
aware of ACM SIGecom, the ACM Special Interest Group on E-commerce.
SIGecom runs an annual conference that includes search as well as
recommendation in the  call for papers, however, for many years search
has not been well represented at this conference.  We will be addressing
this shortfall by examining eCommerce (i.e. product) search from the perspective of search.

This workshop is related to the ESAIR series of workshops (one of our
organisers was on the ESAIR Program Committee from 2010-2016) that
explored semantic annotation - product annotation is commonly seen in eCommerce sites.

Events such as the two SWIRL workshops brought together researchers and
practitioners in Information Retrieval to discuss research agenda.
One of the organisers of this workshop attended both SWIRL events.

\section{Theme and Purpose}

The primary theme of the workshop is product search and
recommendation.  Recall that an output of the workshop will be a
research agenda and as such any accepted papers should further that goal.

The purpose of the workshop is to provide a venue for publication and
discussion of Information Retrieval research and ideas as they pertain
to products and eCommerce.  We will be bringing together practitioners and
researchers from academia and industry to discuss the challenges and
approaches to search and recommendation.  In particular, how
to get data.

A goal is to foster collaboration, discussion in the broader IR
community (facilitated by the published agenda).  We are happy to state
that we have an agenda to raise awareness within the academic community
of the problems faced by this domain.

\subsection{Scope} The workshop relates to all aspects of product search and
recommendations. Research topics and challenges that are
usually encountered in this domain include:

\begin{itemize}
\item Structured data and faceted search, for example, converting unstructured data to its structured form
\item The use of domain specific facets in search and other IR tasks, and how those facets are chosen
\item Temporal dynamics for Search and Recommendation
\item Models for relevance and ranking for multi-faceted entities
\item Deterministic (and other) sorting of results lists (e.g. price low to high including postage)
\item Personalized search and recommendations
\item Inventory display issues (for example: legal, ethical, and spam)
\item Cold start issues
\item Personalization and the use of personal facets such as age, gender, location etc.
\item Indexing and search in a rapidly changing environment (for example, an auction site)
\item Scalability
\item Diversity and whether it is meaningful in product search
\item Strategies for resolving extremely low (or no) recall queries
\item Query intent
\item Semantic understanding of queries
\item The use of external features such as reviews and ratings in ranking
\item User interfaces and personalization
\item The use of social signals in ranking and beyond
\item The balance between business requirements and user requirements (revenue vs relevance)
\item Trust
\item Live experimentation 
\item Desktop and mobile issues
\end{itemize}
\section{Workshop Format}

The workshop will start with an invited talk from a well respected
practitioner (or academic) who is tackling product search problems.
We have contacts in eBay, Amazon, Facebook, Flipkart (amongst others)
but have not approached anyone yet.

We will circulate a call asking not only for
research papers, but also for position and opinion papers.  All
submitted papers and posters will be single-blind, peer reviewed by an
international program committee of researchers of high repute.  Accepted
papers will be presented in slots of 20 minutes with an additional 10
minutes for discussion.  We emphasize that this is not 10 minutes for
questions, but 10 minutes for discussion.  Once the list of accepted
workshops is published we will endeavor to work with organisers of any
other SIGIR 2017 workshops to have a shared (cross-workshop) poster
session (if appropriate). 

After the thought provoking invited talk and presentations we will
break-out into small groups to identify key areas for future research in
product search. Each group will identify one key research problem, endeavour
to identify the challenges and opportunities (which will likely include
the issue of data availability), and report back to the workshop on
their findings.  The ideas that come from this break-out session will
form the basis of the research agenda we will publish.

The final session of the day will be a general discussion session.  The
primary topic for discussion will be data availability.  Specifically,
what guarantees need to be in place for an organisation like eBay to make
a dump available, and can we meet those guarantees?

Our workshop schedule and activities are structured to contain
substantial time for discussion and engagement by all participants.  We
consider this to be a workshop, not a mini-conference.  We are aware
that the session times (and therefore the exact schedule) are usually
set by the SIGIR conference chairs based on catering requirements and
room availability, so we do not have a minute by minute breakdown to
present in this proposal.

\section{Participation}
This workshop is open to all interested parties. Invited speakers will be chosen
by the workshop organisers (who are open to input from the SIGIR organisers). Accepted
papers and posters will be peer-reviewed by members of the program committee who will
make recommendations to the workshop chairs who will make the final decision based on 
space (i.e. the PC chairs are the workshop chairs).

\section{Workshop Outcomes}

We believe that the most important outcome of the workshop is the
discussion between individuals at the workshop.  It is these discussions
that lead to collaboration and future research - unarguably the ultimate
goal of any workshop.  This, however, is hard to directly capture.  We
will capture what we can in the form of a SIGIR Forum workshop report.

We will produce a proceedings of the workshop and work with SIGIR and
ACM to include those proceedings as part of the ACM International
Conference Proceedings Series.  We are aware that this may place a
financial requirement on the workshop organisers and we are currently
exploring funding options.

An important goal is the research agenda.  This will be published as a
separate entity (not part of the SIGIR Forum report) and in a suitable
venue (such as SIGIR Forum).  It will outline research problems the
participants believe to be important.

We expect that there will be lively discussion on data availability.
Since there will be representatives from product sites present, we
will engage them in discussions on the requirements for data release.
Although the release of data is not a specific goal of this workshop, we
expect it to pave the way for such an event in the future.

An additional goal is to raise awareness of the fascinating problems
that litter the product search battlefield.  We hope
that through discussion at the workshop, and SIGIR 2017, we can help
steer the research community towards these problems and in doing so find
solutions to some very difficult problems.

\section{Potential Program Committee}

Below we list potential members of the program committee. It should be
clear that we have included individuals with both experience in
eCommerce, and an academic track record.  We have not contacted any of
the individuals listed here and will not do so unless the workshop is
accepted - this list is simply indicative.

\begin{itemize}
\item Simon Corston-Oliver (Amazon)
\item Mayur Datar (Flipkart)
\item Jon Degenhardt (eBay)
\item Surya Kallumadi (KSU)
\item Mohit Kumar (Flipkart)
\item Ian Soboroff (NIST)
\item Andrew Trotman (University of Otago)
\item Julian McCauley (University of California, San Diego)
\item Jim Jansen (Pennsylvania State University)
\item Daniel Tunkelang (Consultant)
\item Sumit Borar (Myntra Fashion)
\item Chris Severs (Amazon)
\item Tracy King (Amazon)
\end{itemize}

\section{Organizers}
\subsection*{Jon Degenhardt}

\textbf {Jon} has been a senior member of eBay's search science team since 2010.
His primary focus is search relevance, but spends significant time on search
engine architecture and user experience in pursuit of a holistic eCommerce
search experience. Previously he was a member of Yahoo's web search ranking team
(starting 2006). Both experiences have informed his views on the contrasts between
web search and e-commerce search. Jon received CS degrees from the University of
Michigan (BS) and Stanford University (MS) in the 1980s. He has worked in industry
since, including stints in both traditional software engineering and information science
projects. At eBay, he started Applied Science Forum, eBay's most successful
internal data science talk series. He attended his first SIGIR conference in 1987.

\subsection*{Surya Kallumadi}

\textbf{Surya} is a Ph.D. candidate with the Department of Computing and
Information Sciences at Kansas State University. His thesis research is
in the area of Recommendations in Heterogeneous Information Networks. He
has worked in the industry on multiple occasions as a Research Intern at
eBay in the fields of Information Retrieval and structured data. He has
worked with the Data Science team at Flipkart since 2015 in the fields
of Search and Query understanding. Surya has organized 3 Heterogeneous
Information Network Analysis (HINA '11,'13,'15) workshops at IJCAI.

\subsection*{Andrew Trotman}

\textbf{Andrew} gained a B.A. in the 1980s, an M.Sc. in the 1990s, and a Ph.D. in
the 2000s.  He has worked in the field of Information Retrieval since 1992
when he worked in industry designing and implementing one of the first
commercially successful digital libraries (BioMedNet).  He has worked in
industry for a total of 11 years at such establishments as: Elsevier,
PubMed, and eBay.  His academic career spanned 12 years at
the University of Otago (New Zealand) and has seen him as a member of the SIGIR Executive committee (2010-2013),
general chair for SIGIR 2014, INEX (2008-2010), ADCS 2012, program chair for ADCS 2009
and ADCS 2010, and a frequent member of the senior program committee for
top conferences such as SIGIR and CIKM.  Andrew has published over 100
papers (according to Google Scholar), has chaired 6 SIGIR workshops, and
is the designer and primary author of the ATIRE and JASS open source search engines.

\end{document}
